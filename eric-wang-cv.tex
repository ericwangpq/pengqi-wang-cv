\documentclass[10pt, letterpaper]{article}

% Packages:
\usepackage[
    ignoreheadfoot, % set margins without considering header and footer
    top=1 cm, % separation between body and page edge from the top
    bottom=1 cm, % separation between body and page edge from the bottom
    left=2 cm, % separation between body and page edge from the left
    right=2 cm, % separation between body and page edge from the right
    footskip=0.8 cm, % separation between body and footer
    % showframe % for debugging 
]{geometry} % for adjusting page geometry
\usepackage{titlesec} % for customizing section titles
\usepackage{tabularx} % for making tables with fixed width columns
\usepackage{array} % tabularx requires this
\usepackage[dvipsnames]{xcolor} % for coloring text
\definecolor{primaryColor}{RGB}{0, 79, 144} % define primary color
\usepackage{enumitem} % for customizing lists
\usepackage{fontawesome5} % for using icons
\usepackage{amsmath} % for math
\usepackage[
    pdftitle={John Doe's CV},
    pdfauthor={John Doe},
    pdfcreator={LaTeX with RenderCV},
    colorlinks=true,
    urlcolor=primaryColor
]{hyperref} % for links, metadata and bookmarks
\usepackage[pscoord]{eso-pic} % for floating text on the page
\usepackage{calc} % for calculating lengths
\usepackage{bookmark} % for bookmarks
\usepackage{lastpage} % for getting the total number of pages
\usepackage{changepage} % for one column entries (adjustwidth environment)
\usepackage{paracol} % for two and three column entries
\usepackage{ifthen} % for conditional statements
\usepackage{needspace} % for avoiding page brake right after the section title
\usepackage{iftex} % check if engine is pdflatex, xetex or luatex

% Ensure that generate pdf is machine readable/ATS parsable:
\ifPDFTeX
    \input{glyphtounicode}
    \pdfgentounicode=1
    % \usepackage[T1]{fontenc} % this breaks sb2nov
    \usepackage[utf8]{inputenc}
    \usepackage{lmodern}
\fi



% Some settings:
\AtBeginEnvironment{adjustwidth}{\partopsep0pt} % remove space before adjustwidth environment
\pagestyle{empty} % no header or footer
\setcounter{secnumdepth}{0} % no section numbering
\setlength{\parindent}{0pt} % no indentation
\setlength{\topskip}{0pt} % no top skip
\setlength{\columnsep}{0cm} % set column seperation
\makeatletter
\let\ps@customFooterStyle\ps@plain % Copy the plain style to customFooterStyle
\patchcmd{\ps@customFooterStyle}{\thepage}{
    \color{gray}\textit{\small Pg. \thepage{} of \pageref*{LastPage}}
}{}{} % replace number by desired string
\makeatother
\pagestyle{customFooterStyle}

\titleformat{\section}{\needspace{4\baselineskip}\bfseries\large}{}{0pt}{}[\vspace{1pt}\titlerule]

\titlespacing{\section}{
    % left space:
    -1pt
}{
    % top space:
    0.3 cm
}{
    % bottom space:
    0.2 cm
} % section title spacing

\renewcommand\labelitemi{$\circ$} % custom bullet points
\newenvironment{highlights}{
    \begin{itemize}[
        topsep=0.10 cm,
        parsep=0.10 cm,
        partopsep=0pt,
        itemsep=0pt,
        leftmargin=0.4 cm + 10pt
    ]
}{
    \end{itemize}
} % new environment for highlights

\newenvironment{highlightsforbulletentries}{
    \begin{itemize}[
        topsep=0.10 cm,
        parsep=0.10 cm,
        partopsep=0pt,
        itemsep=0pt,
        leftmargin=10pt
    ]
}{
    \end{itemize}
} % new environment for highlights for bullet entries


\newenvironment{onecolentry}{
    \begin{adjustwidth}{
        0.2 cm + 0.00001 cm
    }{
        0.2 cm + 0.00001 cm
    }
}{
    \end{adjustwidth}
} % new environment for one column entries

\newenvironment{twocolentry}[2][]{
    \onecolentry
    \def\secondColumn{#2}
    \setcolumnwidth{\fill, 4.5 cm}
    \begin{paracol}{2}
}{
    \switchcolumn \raggedleft \secondColumn
    \end{paracol}
    \endonecolentry
} % new environment for two column entries

\newenvironment{header}{
    \setlength{\topsep}{0pt}\par\kern\topsep\centering\linespread{1.5}
}{
    \par\kern\topsep
} % new environment for the header

% \newcommand{\placelastupdatedtext}{% \placetextbox{<horizontal pos>}{<vertical pos>}{<stuff>}
%   \AddToShipoutPictureFG*{% Add <stuff> to current page foreground
%     \put(
%         \LenToUnit{\paperwidth-2 cm-0.2 cm+0.05cm},
%         \LenToUnit{\paperheight-1.0 cm}
%     ){\vtop{{\null}\makebox[0pt][c]{
%         \small\color{gray}\textit{Last updated in September 2024}\hspace{\widthof{Last updated in September 2024}}
%     }}}%
%   }%
% }%

% save the original href command in a new command:
\let\hrefWithoutArrow\href

% new command for external links:
\renewcommand{\href}[2]{\hrefWithoutArrow{#1}{\ifthenelse{\equal{#2}{}}{ }{#2 }\raisebox{.15ex}{\footnotesize \faExternalLink*}}}

\renewcommand{\href}[2]{\hrefWithoutArrow{#1}{#2}}

% Define a new counter for references
\newcounter{refcount}

% Define a new environment for IEEE-style references
\newenvironment{ieeerefs}{
    \begin{list}{}{ % Start a list environment
        \setlength{\leftmargin}{10pt} % Adjust left margin as needed
        \setlength{\itemindent}{-4pt} % Negative indentation for hanging format
        \setlength{\labelwidth}{5pt} % Width of the label
        \setlength{\labelsep}{5pt} % No separation between label and item
        \renewcommand{\makelabel}[1]{\hss[##1]\hfil} % Format the label as [n] and align right
    }
}{
    \end{list}
}

% Define a command for each reference item
\newcommand{\ieeeitem}{
    \refstepcounter{refcount} % Increment the reference counter
    \item[\therefcount] % Use the current counter value as the label
}

\begin{document}
    \newcommand{\AND}{\unskip
        \cleaders\copy\ANDbox\hskip\wd\ANDbox
        \ignorespaces
    }
    \newsavebox\ANDbox
    \sbox\ANDbox{}


    % \placelastupdatedtext
    \begin{header}
        \textbf{\fontsize{24 pt}{24 pt}\selectfont Pengqi ``Eric'' WANG}

        \vspace{0.3 cm}

        \normalsize
        \mbox{\hrefWithoutArrow{mailto:wang.19883@osu.edu}{\color{black}{\footnotesize\faEnvelope[regular]}\hspace*{0.13cm}wang.19883@osu.edu}}%
        \kern 0.25 cm%
        \AND%
        \kern 0.25 cm%
        \mbox{\hrefWithoutArrow{https://ericwangpq.github.io/}{\color{black}{\footnotesize\faLink}\hspace*{0.13cm}ericwangpq.github.io}}%
    \end{header}

    \vspace{0.3 cm - 0.3 cm}

    \section{Education}

        \begin{twocolentry}{
        \textit{Aug 2025 - Jul 2030 (Expected)}}
            \textbf{The Ohio State University}

            \textit{Ph. D. in Computer Science and Engineering}
        \end{twocolentry}
        
        \vspace{0.10 cm}
        
        \begin{onecolentry}
            \begin{highlights}
                \item \textbf{Advisor:}  \href{https://www.pingzhang.net/index.html}{Prof. Ping Zhang}, \href{ttps://medicine.osu.edu/find-faculty/non-clinical/biomedical-informatics/weidan-cao}{Prof. Weidan Cao} 
            \end{highlights}
        \end{onecolentry}
       
        \vspace{0.2 cm}

        \begin{twocolentry}{
        \textit{Sep 2023 – Jul 2025}}
            \textbf{The Hong Kong University of Science and Technology}

            \textit{M. Phil. in Computational Media and Arts, Guangzhou campus}
        \end{twocolentry}

        \vspace{0.10 cm}
        \begin{onecolentry}
            \begin{highlights}
                \item \textbf{GPA: 4.06 / 4.3}
                \item \textbf{Cognate: Human Computer Interaction}
                \item \textbf{Thesis Topic:} Enhancing Children's Reading Engagement through Multi-Modal Interactive AI Agent
                \item \textbf{Advisor:} \href{https://www.mingmingfan.com/}{Prof. Mingming FAN (Primary)}, \href{https://sosc.hkust.edu.hk/people/muzhi-zhou/}{Prof. Muzhi ZHOU (Co)} 
                \item \textbf{Awards:} Postgraduate Studentship (\textasciitilde 1,426 USD per month); Postgrad. Research Funding \& Travel Grant
            \end{highlights}
        \end{onecolentry}

        \vspace{0.2 cm}

        \begin{twocolentry}{
        \textit{Sep 2019 – Jun 2023}}
            \textbf{Hong Kong Baptist University}

            \textit{B. B. A. (Honours) in e-Business Management and Information Systems (First Class), with minor in Computer Science and Technology, BNU-HKBU UIC campus}
        \end{twocolentry}

        \vspace{0.10 cm}
        \begin{onecolentry}
            \begin{highlights}
                \item \textbf{GPA: 3.66 / 4.0 (Ranking top 5\%)}
                \item \textbf{Thesis Topic:} Comp. Study of Tree-Based \& Transformer-based Models for Fake Rev. Detection on Yelp
                \item \textbf{Advisor:} \href{https://staff.uic.edu.cn/ywang/en}{Prof. May Ying WANG (Academic)}, \href{https://sites.google.com/view/chaijunyi/home}{Prof. Don Junyi CHAI (Final Year Project, Mentor)}
                \item \textbf{Awards:} First-Class Scholarship (2019-2020), Second-Class Scholarship (2020-2021, 2021-2022, 2022-2023)
            \end{highlights}
        \end{onecolentry}

    \section{Research Interest}

        \begin{onecolentry}
            % Human-AI Collaboration, Educational Technology, Technology for Social Good, Computer-Supported Cooperative Work (CSCW), Assistive Technology, Healthcare, etc.            
            Human-AI Collaboration[1-7, 9-10], Healthcare[2-4, 8], Education Technology[1,6,8], Assistive Technology[5],etc.

        \end{onecolentry}

    \section{Publications}
        Paper in progress:

        \begin{onecolentry}
            \begin{ieeerefs}

                \ieeeitem \textbf{Pengqi Wang}, Kecen Yao, Xiaoying Wei and Mingming Fan*. 2026. PolyTaler: Enhancing Children's Reading Engagement through Multi-Modal Interactive AI Agent. Submitted to \textit{ACM CHI Conference on Human Factors in Computing Systems (CHI '26)}. Under Review. 

                \ieeeitem Bingsheng Yao, Menglin Zhao, Zhan Zhang, \textbf{Pengqi Wang}, Emma G Chester, Changchang Yin, Tianshi Li, Varun Mishra, Lace M. Padilla, Odysseas P Chatzipanagiotou, Timothy Pawlik, Ping Zhang, Weidan Cao, Dakuo Wang*. 2026. Exploring Collaboration Breakdowns Between Provider Teams and Patients in Post-Surgery Care. Submitted to \textit{ACM CHI Conference on Human Factors in Computing Systems (CHI '26)}. Under Review. \href{https://arxiv.org/abs/2509.23509}{arxiv}

                \ieeeitem Xianhui Chen, \textbf{Pengqi Wang}, Changchang Yin, Ping Zhang*. 2026. Multi-Agent Retrieval for Clinical Trial Outcome Prediction. Submitted to \textit{the ACM Web Conference 2026 (WWW '26)}. Under Review. \href{https://main.dojc6iq05ocao.amplifyapp.com/}{demo}

                
                \ieeeitem Xianhui Chen\dag, Changchang Yin\dag, \textbf{Pengqi Wang}, Weidan Cao, Ping Zhang*. 2026. Smartphone-Based Parkinson's Disease Detection with Active Sensing. Submitted to \textit{the 32st ACM SIGKDD Conference on Knowledge Discovery and Data Mining (KDD'26)}. Under Review. \href{https://github.com/XianhuiChen/PDSensing}{code \& iOS App demo}

                


                \ieeeitem Beiyan Cao, \textbf{Pengqi Wang}, Qianjie Wei, Changyang He, Jianyuan Wang, Muzhi Zhou, Mingming Fan*. 2026. Exploring the Design of AI-mediated Emotion Communication for Deaf and Hard of Hearing People in Online Meetings. Submitted to CSCW '26.

                \ieeeitem \textbf{Pengqi Wang}, Mingqing Xu, Li Feng and Mingming Fan*. 2026. Student Learning with AI in Multi-Source Information Environments. Submitted to CHI '25, Round 2 Reviewed. \href{https://ericwangpq.github.io/files/PublishedPapers/a0-CHI25_in_depth_qualitative_study_LLM_powered_Learning.pdf}{paper}



            \end{ieeerefs}
        \end{onecolentry}
        
        \vspace{2mm} Published paper:
        \begin{onecolentry}
            \begin{ieeerefs}

                


                
                \ieeeitem Yuanlong Wang, \textbf{Pengqi Wang}, Changchang Yin, Ping Zhang*. 2025. SatHealth: A Multimodal Public Health Dataset with Satellite-based Environmental Factors. In \textit{Proceedings of the 31st ACM SIGKDD Conference on Knowledge Discovery and Data Mining (KDD'25)}. \href{https://dl.acm.org/doi/10.1145/3711896.3737440}{paper}; \href{https://aimed-sathealth.net/}{demo}
                
                \ieeeitem Qianjie Wei, Jingling Zhang, \textbf{Pengqi Wang}, Xiaofu Jin and Mingming Fan*. 2024. Augmented Library: Toward Enriching Physical Library Experience Using HMD-Based Augmented Reality. In \textit{Proc. 17th Int. Symp. Visual Information Communication and Interaction (VINCI '24)}. \href{https://dl.acm.org/doi/abs/10.1145/3678698.3687174}{paper}
                % \ieeeitem May Ying Wang, Bingjie Deng* and \textbf{Pengqi Wang}*. 2024. From Hallucination to Trust: Leveraging Chain-of-Thoughts and Empathy in LLM-Empowered Agent. In \textit{Proc. 24th Int. Conf. Electronic Business (ICEB '24)}. \href{https://ericwangpq.github.io/files/PublishedPapers/a2-FromHallucination.pdf}{paper}
                
                % \ieeeitem May Ying Wang, Na Jiang, \textbf{Pengqi Wang} and Bingqing Xiong. 2024. Enhancing Well-Being and Reliance on AIGC-Powered Digital Assistants. In \textit{2024 UNNC-CNAIS Paper Development Workshop}.
                
                \ieeeitem May Ying Wang and \textbf{Pengqi Wang}*. 2023. Decoding Business Applications of Generative AI: A Bibliometric Analysis and Text Mining Approach. In \textit{Proc. 23rd Int. Conf. Electronic Business (ICEB '23)}. \href{https://ericwangpq.github.io/files/PublishedPapers/a5-DecodingBusinessApplications.pdf}{paper}
                \ieeeitem \textbf{Pengqi Wang}, Yue Lin and Junyi Chai*. 2023. Unmasking Deception: A Comparative Study of Tree-Based and Transformer-based Models for Fake Review Detection on Yelp. In \textit{Proc. IEEE Int. Conf. Systems, Man, and Cybernetics (SMC '23)}. \href{https://ericwangpq.github.io/files/PublishedPapers/a6-Unmasking_Deception.pdf}{paper}
                \ieeeitem \textbf{Pengqi Wang}, Mengli Yu and Yadong Liu*. 2022. Assessing the Content Topics of the Educational Videos on Tik Tok for Science Communication. In \textit{Proc. Int. Seminar on Education, Management and Social Sciences (ISEMSS '22)}. \href{https://ericwangpq.github.io/files/PublishedPapers/a8-AssessingContent.pdf}{paper}
            \end{ieeerefs}
        \end{onecolentry}

% \newpage
% ===============================
% Selected Research Experiences
% ===============================

    \section{Selected Research Experiences}

        \begin{twocolentry}{
        \textit{Aug 2025 - Present}}
            \textbf{Graduate Research Associate} 
    \end{twocolentry}
    \hspace{0.2cm}\textit{\href{https://www.pingzhang.net/lab.html}{AIMed} Lab}, OSU (Supervised by Prof. Ping Zhang and Prof. Weidan Cao)


    \vspace{0.10 cm}
    \begin{onecolentry}
        \begin{highlights}
            \item \textbf{From Hallucination to Trust: Leveraging CoT and Empathy in LLM-Empowered Agent}
            \begin{itemize}
                \item Investigated the enhancement of user experience with LLMs in critical scenarios, such as hallucination and service failure, through Chain-of-Thought reasoning and empathy expressions. Contributed to developing trustworthy and user-centric AI systems.
                \item Developed an AI-powered conversational agent based on the LobeHub framework for a field study experiment. Prepared the questionnaire and authored the manuscript.
            \end{itemize}
        \end{highlights}
    \end{onecolentry}
    


% ===============================
% Research Student @ HKUST
% ===============================

        \begin{twocolentry}{
        \textit{Sep 2023 – Jul 2025}}
            \textbf{Research Student}
        \end{twocolentry}
        \hspace{0.2cm}\textit{\textbf{A}ccessible \& \textbf{P}ervasive User \textbf{EX}perience (\href{https://www.mingmingfan.com/lab/}{APEX}) Lab}, HKUST (Supervised by Prof. Mingming FAN)


        \vspace{0.10 cm}
        \begin{onecolentry}
            \begin{highlights}
                \item \textbf{Enhancing Children’s Reading Engagement through Multi-Modal Interactive AI Agent}
                \begin{itemize}
                    \item Designed and implemented \textbf{multi-modal reading companion system} integrating image generation, emotion recognition, adaptive conversational reasoning, and natural voice synthesis to foster engaging reading experiences.
                \end{itemize}
                \item \textbf{Understanding and Facilitating Learning with AI in Multi-Source Information Environment for College Students}
                \begin{itemize}
                    \item Led qualitative research on AI-powered tools for college students. The study focused on multi-source information environments. It revealed a framework of students' information behaviors and provided \textbf{design implications} to enhance AI-assisted academic tools.
                    \item Conducted \textbf{focus group study sessions} and \textbf{participatory design workshops}, analyzed the data using \textbf{inductive and deductive analysis} to generate actionable insights, and authored the manuscript.
                \end{itemize}
                % \item \textbf{Augmented Library: Toward Enriching Physical Library Experience Using HMD-Based AR}
                % \begin{itemize}
                %     \item Designed and developed an HMD-based \textbf{AR} system, Augmented Library. This system aims to revitalize physical library experiences for college students. Integrated interactive digital features to enhance book discovery and foster community engagement within the library setting.
                % \end{itemize}
                \item \textbf{Exploring the Design of AI-mediated Emotion Communication for Deaf and Hard of Hearing People in Online Meetings}
                \begin{itemize}
                    \item Designed, developed, and implemented an emotion support system for \textbf{Deaf and Hard of Hearing (DHH)} participants in online meetings. The system detects, interprets, and visualizes emotional states in real-time, addressing unique emotional challenges faced by DHH users in virtual environments.
                    \item Collaborated with the DHH community for user evaluations, assessing the system's effectiveness in improving participation and reducing social isolation.
                \end{itemize}
                
                % \item \textbf{Investigating the Role of AI in the Team Synchronous-Asynchronous Collaboration Loop}
                % \begin{itemize}
                %     \item Proposed and developed an AI-assisted collaborative ideation system to enhance team collaboration across synchronous and asynchronous modes. Focused on improving idea comprehension and evolution in group ideation processes. Prepared and conducted user evaluations to assess the effectiveness of AI-assisted features in collaborative environments.
                % \end{itemize}
                
            \end{highlights}
        \end{onecolentry}

% ===============================
% Research Student @ NKU - uncomment this only for health care related research
% ===============================

% \begin{twocolentry}{
%     \textit{Jun 2024 – Sep 2024}}
%         \textbf{Summer Research Assistant}   
% \end{twocolentry}
% \hspace{0.2cm} Nankai University (Supervised by Prof. Mengli YU)
% \begin{onecolentry}
%     \begin{highlights}
%         \item \textbf{Advancing Trustworthiness of AI-Powered Chatbot Applied in
%         Virtual Triage}
%         \begin{itemize}
%             \item Proposed and developed an AI-assisted chatbot to support the triage process in hospitals by providing multi-run guided dialogues, references to reliable and verifiable information, and empathetic responses to enhance user trust.
%         \end{itemize}
%     \end{highlights}
% \end{onecolentry}

% ===============================
% Research Assistant @ HKBU-UIC
% ===============================


        % \begin{twocolentry}{
        %     \textit{May 2023 – Sep 2023}}
        %         \textbf{Research Assistant}   
        % \end{twocolentry}
        % \hspace{0.2cm}\textit{Faculty of Business and Management}, HKBU-UIC (Supervised by Prof. May Ying WANG)


        % \vspace{0.10 cm}
        % \begin{onecolentry}
        %     \begin{highlights}
        %         \item \textbf{From Hallucination to Trust: Leveraging CoT and Empathy in LLM-Empowered Agent}
        %         \begin{itemize}
        %             \item Investigated the enhancement of user experience with LLMs in critical scenarios, such as hallucination and service failure, through Chain-of-Thought reasoning and empathy expressions. Contributed to developing trustworthy and user-centric AI systems.
        %             \item Developed an AI-powered conversational agent based on the LobeHub framework for a field study experiment. Prepared the questionnaire and authored the manuscript.
        %         \end{itemize}
        %     \end{highlights}
        % \end{onecolentry}

        % \vspace{0.10 cm}
        % \begin{onecolentry}
        %     \begin{highlights}
        %         \item \textbf{Enhancing Well-Being and Reliance on AIGC-Powered Digital Assistants}
        %         \begin{itemize}
        %             \item Investigated the impact of explainability signals and dialogue strategies on user well-being and reliance on AIGC-powered digital assistants. Aimed to refine their design and enhance user satisfaction.
        %             \item Designed experiments and developed systems. Presented the working paper at the 2024 UNNC-CNAIS Paper Development Workshop, gathering valuable insights from conference mentor Atreyi Kankanhalli for future research directions.
        %         \end{itemize}
        %     \end{highlights}
        % \end{onecolentry}

    \section{Professional Experiences}

        \begin{twocolentry}{
        \textit{Jul 2022 – Nov 2022}}
            \textbf{Data Analyst Intern}

            \textit{eBay Inc. @ Top Seller Account Management Team}
        \end{twocolentry}
        \begin{onecolentry}
            \begin{highlights}
                Analyzed and visualized data using \textbf{SQL}, \textbf{Python}, \textbf{R}, \textbf{Excel VBA}, and \textbf{Tableau} to create interactive dashboards that trended key business indicators, supported data analysis for marketing growth by 40\%, and enhanced decision-making for 1k+ top sellers.
            \end{highlights}
        \end{onecolentry}

        % \vspace{0.2 cm}

        % \begin{twocolentry}{
        % \textit{Jun 2022 – Jul 2022}}
        %     \textbf{Information Systems Intern}

        %     \textit{Kingdee International Software Group @ Kingdee China Shared Service Center}
        % \end{twocolentry}
        % \begin{onecolentry}
        %     \begin{highlights}
        %         Maintained and optimized major info. sys. (ERP, CRM, KBC), processed over 100 issues by providing remote \textbf{SQL} and \textbf{Python} support, and enhanced the efficiency of Intelligent Approval System by 20\%.
        %     \end{highlights}
        % \end{onecolentry}

    % \section{References}

    %     \begin{twocolentry}{
    %         \textit{mingmingfan@ust.hk}}
    %             \textbf{Prof. Mingming FAN}
    %     \end{twocolentry}
    %     \begin{onecolentry}
    %         Asst. Prof. in Computational Media \& Arts and Internet of Things at HKUST (Guangzhou); \\
    %         Asst. Prof. in Div. of Integrative Systems \& Design and Dept. of Comp. Sci. \& Engineering at HKUST.
    %     \end{onecolentry}

    %     \vspace{0.2 cm}

    %     \begin{twocolentry}{
    %         \textit{ywang@uic.edu.cn}}
    %             \textbf{Prof. May Ying WANG}
    %     \end{twocolentry}
    %     \begin{onecolentry}
    %         Assoc. Prof., Programme Director, Faculty of Business and Management, Beijing Normal University-Hong Kong Baptist University United International College.
    %     \end{onecolentry}

    \section{Skills}

        \begin{onecolentry}
            \begin{highlights}                
                % \item \textbf{Languages:} English \textit{(\textbf{TOEFL iBT 102, Best Score: 106})}: W 27 \textbf{(27)}, R 30 \textbf{(30)}, L 22 \textbf{(26)}, S 23 \textbf{(23)}, Mandarin (Native).
                \item \textbf{English Proficiency:} TOEFL iBT 102 (\textbf{Best Score: 106}): W 27 \textbf{(27)}, R 30 \textbf{(30)}, L 22 \textbf{(26)}, S 23 \textbf{(23)}.
                \item \textbf{Programming \& Quantitative Analysis:} Python (NumPy, Pandas, Scikit-learn, PyTorch, etc.), SQL, R, C/C++, JavaScript, HTML/CSS, LaTeX.
                \item \textbf{Developer Tools:} Docker, AWS (EC2, Lightsail, Bedrock, etc.), Azure, Git, Linux.
                \item \textbf{Qualitative Analysis:} Inductive and Deductive Analysis, Thematic Analysis, Grounded Theory; Atlas.ti, NVivo, MAXQDA.
                \item \textbf{UX/UI \& Creativity:} Figma, Adobe Kit (Photoshop, Illustrator, Premiere Pro).
                \item \textbf{Research Methodologies:} Focus Group, Participatory Design, Field Study, Literature Review.
                \item \textbf{Other:} Large Language Models (LLMs), GenAI, Facial Emotional Recognition (FER), Chatbot, VR/AR/XR.
            \end{highlights}
        \end{onecolentry}

\end{document}